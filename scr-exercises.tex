\documentclass{article}

\usepackage{amsmath}
\usepackage[natbibapa,nosectionbib]{apacite}
\usepackage{enumitem}
\usepackage[textwidth = 6in, top = 2cm, bottom = 2cm]{geometry}
\usepackage{parskip}
\usepackage{xparse}
\DeclareDocumentCommand{\sct}{ O{} O{} m }{\shortcites{#3}\citet[#1][#2]{#3}}

\title{Simple SCR exercises}
\date{}

\begin{document}

\maketitle

The code in the file \texttt{scr-ll.r} creates a log-likelihood
function, loads some example data, and fits the `binary proximity
model' described by \sct{Efford2009}. The model makes the following
assumptions:
\begin{itemize}
\item The number of animals' activity centres in the survey region is
  a Poisson random variable, with expectation equal to animal density,
  $D$, multiplied by the area of the survey region.
\item The activity centre locations are independent, and are uniformly
  distributed across the survey region.
\item The probability that a detector detects an individual is given
  by the halfnormal detection function,
  \begin{equation*}
    g(d) = g_0 \exp\left( \frac{-d^2}{2\sigma^2} \right),
  \end{equation*}
  where $d$ is the distance between the animal's activity centre and
  the detector.
\end{itemize}

Run the code in \texttt{scr-ll.r} and answer the following questions.

\begin{enumerate}

\section*{General questions}
  
\item Creat a plot of the detector locations. Their coodinates can be
  found in \texttt{test.data\$traps}.
  
\item Inspect the capture histories in
  \texttt{test.data\$bin.capt}. Describe what the first two rows
  represent.

\item The model has estimated animal density, $D$, and parameters of a
  halfnormal detection function, $g_0$ and $\sigma$. Create a plot of
  the detection function estimated by the model.

\item Tricky question for STATS 730 graduates only:
  \begin{enumerate}
  \item[(a)] Compute standard errors for the transformed parameters,
    $\log(D)$, $\text{logit}(g_0)$, and $\log(\sigma)$. The
    \texttt{optim()} argument \texttt{hessian} will be useful.
  \item[(b)] Compute standard errors for the parameters themselves, $D$,
    $g_0$, and $\sigma$.
  \item[(c)] Compute confidence intervals for the three parameters.
  \end{enumerate}
  
\item Write some R code that simulates capture histories from a
  spatial capture-recapture model under the following conditions:
  \begin{itemize}
  \item The survey region is a square, with x-coordinate limits
    $(-500, 900)$ and y-coordinate limits also $(-500, 900)$. Note
    that these coordinates are given in metres.
  \item Detectors are deployed on a three-by-three grid with a $100$ m
    spacing between them, so that the columns are located at
    x-coordinates $100$, $200$, and $300$, and the rows are located at
    y-coordinates $100$, $200$, and $300$. Note that this is the
    configuration of the detectors in \texttt{test.data\$traps}.
  \item Animal density is $D = 0.75$ animals per hectare. Note that
    $1$ hectare is $10\,000$ m$^2$.
  \item Conditional on its activity centre location, an individual is
    detected by a detector with probability given by a halfnormal
    detection function with $g_0 = 0.9$ and $\sigma = 75$ m.
  \end{itemize}

  For bonus points, write your R code as a function, allowing the user
  to set their own detector locations and parameter values.

\item Fit a spatial capture-recapture model to your simulated
  data. Note that you can use the detector locations in
  \texttt{test.data\$traps} and the mask in
  \texttt{test.data\$mask}. How close are your estimates to the true
  parameter values?

\item For STATS 730 graduates only: Compute confidence intervals for
  the three parameters. Did they capture the true parameter values?

\item Run a simulation study, repeating Questions 5--7 a total of 100
  times. This involves simulating 100 sets of capture histories, and
  generating estimates from each. Inspect your 100 sets of estimates.
  \begin{enumerate}
  \item[(a)] How close are the averages of your parameter estimates to
    the true parameter values?
  \item[(b)] For STATS 730 graduates only: How often do your confidence
    intervals capture the true parameter values?
  \end{enumerate}

  \section*{Questions for \texttt{secr} users}

\item Fit the same model from \texttt{secr-ll.r}, but using the
  \texttt{secr} package. Verify that you get the same parameter
  estimates. For STATS 730 graduates, also verify that you get the
  same standard errors. This will require some data reformatting.

  \section*{Questions for \texttt{ascr} users}

\item Fit the same model from \texttt{secr-ll.r}, but using the
  \texttt{ascr} package. Verify that you get the same parameter
  estimates. For STATS 730 graduates, also verify that you get the
  same standard errors.
  
\end{enumerate}

\bibliographystyle{tcite}
\bibliography{$HOME/mega/bib/refs}%$

\end{document}